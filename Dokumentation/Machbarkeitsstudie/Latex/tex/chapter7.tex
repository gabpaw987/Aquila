\chapter{Management Summary} \label{chapter:management_summary}

Nach ausf�hrlicher Besch�ftigung mit den kritischen Themen sind f�r das Projekt AQUILA geeignete L�sungswege entwickelt worden und es wurde aus mehreren potentiellen Varianten die g�nstigste ausgew�hlt. Folglich ist das Projekt durchf�hrbar.\\
\\
Aus der Variantenbildung ergab sich die Wahl der Programmiersprachenkombination C\#/F\# f�r die Software, da dadurch die Performance der Programmierung und der Ausf�hrung gleicherma�en gegeben ist. Die Webschnittstelle soll in ASP.NET implementiert werden, da dadurch die Funktionen der .NET-API weitergehend verwendet werden k�nnen. Weil beide Technologien aus einer Hand kommen, kann \gls{wcf} f�r die Schnittstelle zwischen beiden Komponenten verwendet werden.\\
\\
Die Gesamtkosten des Projektes wurden auf \EUR{44381} veranschlagt. Empfohlene Preise f�r die Software liegen bei \EUR{1999} f�r eine Einzelbenutzerlizenz, \EUR{4999} f�r eine 5-Benuter Lizenz und \EUR{9999} f�r eine unbeschr�nkte Lizenz.\\
\\
Der gesch�tzte Aufwand des Projektes betr�gt 213.3 Stunden, der auf 3 Projektteammitglieder aufgeteilt wird. Der geplante Projektzeitraum ist vom 14.11.2012 bis 10.04.2013.
	
\chapter{Versionierung} \label{chapter:versionierung}
\begin{center}

\begin{tabular}{ | l | p{2cm} | p{2cm} | p{2cm} | p{1.5cm} | p{2.8cm} | }
\hline 
\textbf{Version} & \textbf{Autor} & \textbf{QS} & \textbf{Datum} & \textbf{Status} & \textbf{Kommentar} \\  \hline
0.1 & Pawlowsky & Sochovsky & 22.09.2012 & draft & Erstversion \\ \hline
0.2 & Sochovsky & Pawlowsky & 26.09.2012 & draft & Technische Machbarkeit \\ \hline
0.3 & Pawlowsky & Nagy & 27.09.2012 & draft & Produkt-funktionen\\ \hline
0.4 & Nagy & Sochovsky & 27.09.2012 & draft & IST-Erhebung und Zustand\\ \hline
0.5 & Pawlowsky & Sochovsky & 28.09.2012 & draft & FPA \\ \hline
0.6 & Nagy & Pawlowsky & 29.09.2012 & draft & Kosten/Nutzen \\ \hline
0.7 & Sochovsky & Pawlowsky & 06.10.2012 & draft & Projekt-organisation \\ \hline
0.8 & Pawlowsky & Nagy & 07.10.2012 & draft & Aufwands-absch�tzung \\ \hline
0.9 & Sochovsky & Nagy & 10.10.2012 & draft & Sollzustand \\ \hline
1.0 & Nagy & Sochovsky & 17.10.2012 & draft & Management-Summary, Einleitung \\ \hline
\end{tabular}

\end{center}