\chapter{Globale Testf�lle}
\label{chapter:Globale_Testf�lle}

/T010/ \textbf{Angabe eines falschen Accounts} \\
Wenn ein Benutzer auf der Website versucht, sich mit einem nicht vorhandenen Benutzernamen oder einem falschen Passwort anzumelden, soll es ihm nicht m�glich sein, weitere Einstellungen oder Ver�nderungen am System vorzunehmen.\\
\\
/T020/ \textbf{B�rse hat geschlossen} \\
Wenn der Fall eintritt, dass die B�rse geschlossen ist, soll weiterhin die M�glichkeit bestehen, auf der Website Einstellungen und Ver�nderungen vorzunehmen; wenn diese die Software betreffen, wird diese, die �nderungen �bernehmen und bei Ablaufstart anwenden.\\
\\
/T040/ \textbf{Aktie entfernen}\\
Eine Aktie, mit der gehandelt wird, soll �ber die Website aus dem Handelsumfang der Software einfach und schnell entfernt werden k�nnen. Es ist wichtig, dass diese Funktion sehr schnell abl�uft, damit diese Aktie nicht l�nger gehandelt wird.\\
\\
/T040/ \textbf{Empfangen der Einstellungen von der Website} \\
Wenn die Website Befehle erh�lt, die Einstellungen der Software zu �ndern, sollen diese �nderungen an die Software �bertragen und dann �bernommen werden. Dies soll im Logfile ersichtlich sein.\\
\\
/T050/ \textbf{Aktie hinzuf�gen} \\
Es soll auf der Website m�glich sein, eine Aktie hinzuzuf�gen mit der gehandelt wird. Diese Einstellung soll �ber die Verbindung mit der Software direkt �bergeben werden und sofort in Kraft treten. Diese Interaktion ist essentiell, weil man nicht lange Wartezeiten in Kauf nehmen will, bevor eine neue Aktie hinzugef�gt werden kann.\\
\\
/T060/ \textbf{Daten empfangen, Log bearbeiten und Entscheidung speichern} \\
Nach dem Start der Software sollen Daten empfangen werden, welche dann in den Algorithmus eingespeist werden, um mit ihnen zu rechnen. Das Ergebnis dieses Prozesses ist eine Entscheidung. Diese kann man im Log, in der Datenbank oder auch auf der Website nachvollziehen.\\
\\
/T070/ \textbf{Charts anzeigen} \\
Auf der Website soll es m�glich sein, Charts zu den letzten Berechnungen zu sehen. Zus�tzlich kann man auf der gleichen Seite, auf der sich die Charts befinden, auch die letzte Entscheidung des Algorithmus betrachten. Dadurch erkennt man die Konnektivit�t zwischen Website und Software.\\