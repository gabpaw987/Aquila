% !TeX root = ../Aquila_Machbarkeitsstudie.tex
% Chapter1

\chapter{Einleitung} \label{chapter:einleitung}

Aktienhandel ist meist mit viel Erfahrung verbunden. Umso kurzfristiger gehandelt wird, desto mehr Konzentration und Aufmerksamkeit muss den Vorg�ngen gewidmet werden, damit auch bei mehreren Trades am Tag summa summarum eine positive Bilanz entsteht. Dies macht es f�r kleine Firmen schwierig und f�r handelnde Privatpersonen nahezu unm�glich in diesem Zeitraum zu operieren. Das Ziel dieses Projektes ist es, genau bei dieser Zielgruppe zu punkten, indem eine Software zur algorithmischen Abbildung eines Handelssystems geschaffen wird, die automatisch Kauf- und Verkaufentscheidungen trifft. Eine Website als Schnittstelle zum Benutzer gew�hrleistet Plattformunabh�ngigkeit, gibt Informationen �ber Performance und Preisentwicklungen und erm�glicht das �ndern von Parametern von �berall. Zus�tzlich k�nnen mehrere Benutzer von unterschiedlichen Standorten mit einem gemeinsamen Portfolio operieren.\\
	Charts auf der Website bieten live einen �berblick �ber die aktuelle Situation und dadurch eine erh�hte Transparenz der Arbeitsweise und Entscheidungsgenerierung des Algorithmus. Relevante Parameter, wie die zu handelnden Aktien oder die H�he der Investition k�nnen ebenfalls einfach �ber die Webschnittstelle ver�ndert werden.