% !TeX root = ../Aquila_Pflichtenheft.tex

% Chapter2: Produkteinsatz

\chapter{Produkteinsatz} \label{chapter:produkteinsatz}
\section{Anwendungsbereiche und Zielgruppen}

Das Produkt implementiert ein Handelssystem f�r Aktien. Daher ist f�r einen vern�nftigen Umgang mit der Software ein Mindestwissen �ber Aktienhandel und B�rsengesch�fte vorauszusetzen. Vorteilhaft w�re ebenso ein Verst�ndnis von technischer Analyse und h�ufig genutzten Indikatoren, um darauf basierende Handelssysteme zu durchschauen. Programmier- b.z.w. vertiefende Computerkenntnisse m�ssen hingegen nicht vorhanden sein.\\
	Als Zielgruppe sind insbesondere kleine und mittlere Unternehmen (KMU) anvisiert, in denen sich bereits Personen mit finanzwirtschaftlichen Angelegenheiten befassen. Hinzu kommen private Einzelpersonen, die �ber das n�tige Kapital f�r kurzfristigen Aktienhandel verf�gen und mit wenig Aufwand ein Komplettsystem dazu anwenden m�chten.

\section{Betriebsbedingungen}

Die Software soll es prinzipiell erm�glichen unbeaufsichtigt, selbstst�ndig zu arbeiten, wobei, da es sich um einen substanziellen Kapitalaufwand handeln kann, es insbesondere in der Anfangszeit ratsam ist, die Abl�ufe der Software zu �berwachen.\\
	Sowohl die Handelssoftware selbst, als auch die Webseite laufen auf einem Server. Beide Komponenten m�ssen die Aufteilung auf und somit die Kommunikation zwischen mehrere Server nicht unbedingt unterst�tzen, jedoch w�re dies vorteilhaft, da sowohl Zugriffsrechte separat geregelt werden k�nnten, sowie die Handelssoftware die vollen Kapazit�ten des Servers nutzen kann.\\
	Die Software und die Website soll einen Betrieb rund um die Uhr erm�glichen, ob ein solcher im Einsatz tats�chlich realisiert wird, liegt am Kunden.