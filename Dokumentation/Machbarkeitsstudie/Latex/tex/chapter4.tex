\chapter{Technische Machbarkeit}\label{chapter:technische_machbarkeit}

\section{Programmiersprachen}

Die Tradingsoftware kann man in jeder erdenklichen Programmiersprache schreiben, allerdings ist es wichtig daran zu denken, dass das Programm einerseits effizient arbeiten soll und deswegen hardwarenahe rechnet, und andererseits hat das Projektteam mit manchen Programmiersprachen keinerlei Erfahrung.\\
Die allgemeine Funktionalit�t muss das Einlesen von Echtzeitdaten der B�rse und das Berechnen der Kaufentscheidung sein. F�r das Team kommen daher 3 M�glichkeiten in Frage: Eine L�sung in reinem C++, welches sehr hardwarenahe arbeitet, eine Mischung aus F\# und C\#, mit der eine parallelisierte Berechnung m�glich w�re, und eine reine F\#-L�sung. \\
Bei der Kombination agiert C\# als Handlungs- und Steuerkern, als auch zur Kommunikation mit der Website und dem News-Feed und F\# als funktionale Programmiersprache, als Rechenkern und \"Mastermind\" der Applikation, welche die Entscheidungen trifft. Hierbei wird einerseits eine enorm hohe Arbeitsgeschwindigkeit erm�glicht, da die beiden Sprachen relativ hardwarenah agieren und andererseits besteht der nicht zu untersch�tzende Vorteil bzw. die M�glichkeit, den Rechenkern auf ein externes System outzusourcen, welches zum Beispiel enorme Rechenkapazit�ten aufweisen k�nnte und somit viel komplexere und effizientere Algorithmen in annehmbarer Zeit durchrechnen und abh�ngig davon mehr gewinnbringende Entscheidungen treffen k�nnte. Dabei sollte es auch bei sp�teren Erweiterungen des Programms zu keinem signifikanten Geschwindigkeitsabfall kommen.


\begin{center}

\begin{tabular}{ | c | p{2.6cm} | p{1.5cm} | p{0.5cm} |p{0.5cm}|p{0.5cm}|p{0.5cm}|p{0.7cm}|p{0.7cm}|}
\hline 
\multicolumn{2}{|p{1.5cm}|}{ }  & Ge\-wicht\-ung & \multicolumn{2}{p{1.5cm}|}{\textbf{C++ R*G}} & \multicolumn{2}{p{1.5cm}|}{\textbf{F\# R*G}} & \multicolumn{2}{|p{1.5cm}|}{\textbf{C\#F\# R*G}}\\ \hline
\multirow{6}{*}{Einfachheit} & Aufwand Coding & 10\% & 3 & 30 & 2 & 20 & 1 & 10 \\ \cline{2-9}
& Bedienung/ Wartung & 6\% &3&9&2&6&1&3\\ \cline{2-9}
& Update &3\%&3&9&2&6&1&3\\ \cline{2-9}
& Integration &5\%&3&15&2&10&1&5\\  \cline{2-9}
& Kenntnisse &6\%&3&18&2&12&1&6\\ \cline{2-9}
& \textbf{Gesamt}&30\%&3&81&2&54&1&27\\ \hline
\multirow{5}{*}{Leistung}& �ber\-tragungs-zeit &6\%&1&6&3&18&2&12\\ \cline{2-9}
& Absturz\-sicherheit &5\%&1&5&2&10&3&15\\ \cline{2-9}
& Ressourcen-verbrauch &3\%&1&3&3&9&2&6\\ \cline{2-9}
& Datenumfang &1\%&1&1&3&3&2&2\\ \cline{2-9}
& \textbf{Gesamt} &15\%&1&15&3&40&2&35\\ \hline
\multirow{5}{*}{Kosten}& Lizenzen &10\%&1&10&1&10&1&10\\ \cline{2-9}
& Support &5\%&3&15&1&5&2&10\\ \cline{2-9}
& Betriebs-kosten &5\%&2&10&3&15&1&5\\ \cline{2-9}
& Dokumen\-tation &5\%&1&5&3&15&2&10\\ \cline{2-9}
& \textbf{Gesamt} &15\%&2&40&3&45&1&35\\ \hline
\multirow{4}{*}{Dokumentation}& Verf�gbarkeit &10\%&2&20&3&30&1&10\\ \cline{2-9}
& Voll\-st�ndigkeit &10\%&3&30&2&20&1&10\\ \cline{2-9}
& Qualit�t &10\%&2&20&3&30&1&10\\ \cline{2-9}
& \textbf{Gesamt} &30\%&3&80&3&80&1&30\\ \hline
\end{tabular}

\end{center} 

\begin{center}

\begin{tabular}{|l|r|p{0.8cm}|p{0.8cm}|p{0.8cm}|p{0.8cm}|p{0.8cm}|p{0.8cm}|} \hline
Kapitel&Gewichtung&\multicolumn{2}{p{1.8cm}|}{\textbf{C++}}&\multicolumn{2}{p{1.8cm}|}{\textbf{F\#}}&\multicolumn{2}{p{1.8cm}|}{\textbf{C\#/F\#}}\\ \hline
Einfachheit&30\%&3&81&2&54&1&27 \\ \hline
Leistung&15\%&1&15&3&40&2&35 \\ \hline
Kosten&15\%&2&40&2&45&1&35 \\ \hline
Dokumentation&30\%&3&80&3&80&1&30 \\ \hline
\end{tabular}

\end{center}

\begin{center}
\begin{tabular}{|l|l|l|l|} \hline
Gesamtbewertung&&&\\ \hline
Endreihung &2&3&1\\ \hline
\end{tabular}
\end{center}

Aus der Nutzwertanalyse kann man entnehmen, dass die C\#/F\# Kombination als die favorisierte M�glichkeit ausgeht, weitere Vorteile die sich aus der Wahl dieser Mischung ergeben sind: gute Kenntnisse der Programmiersprachen, tolle Community und die Einfachheit, sowie die Erweiterbarkeit. Bei dieser L�sung wird die Steuereinheit vom C\# Teil des Programms �bernommen und die Rechenaufgaben werden von dem F\# Teil bearbeitet. Au�erdem ist das .net-Framework sehr beliebt, deswegen kann man damit rechnen das bei einem Problem gen�gend Helfer gefunden werden k�nnen.

\section{Websprachen}

Bei der Wahl der Technologie zur Umsetzung der Web-Controlling-Oberfl�che m�ssen auf einige Kriterien geachtet werden, um zu einer Entscheidung zu kommen.
Die Sprache muss es erm�glichen, eventuell durch API-Zugriff, hinreichend komplexe Charts zu generieren, um alle gew�nschten Indikatoren und Preisentwicklungen nahe der gew�nschten Form darzustellen.
Au�erdem muss ein Zugriff auf eine mit der Software gemeinen Datenbank m�glich sein, um Kurswerte und eventuell andere berechneten Daten zur Darstellung abzufragen.\\
	Zur Durchf�hrung der Webfunktionalit�t kommen folgende Sprachen in die engere Auswahl.

\begin{itemize}
	\item PHP
	\item ASP.NET
	\item Java Servlet mit JSP
\end{itemize}

Der augenscheinliche Vorteil von PHP ist die bereits gesammelte Erfahrung des Projektteams mit der Sprache. Ansonsten sind die Chartingfunktionen beispielsweise durch die pChart Bibliothek bereitgestellt. \footnote{url{http://www.pchart.net/}} pChart bietet verschiedene Varianten zur Darstellung an. Unter anderem Line-Charts, Candle-Charts und Indikatoren.\\
\\
ASP.NET bietet den Vorteil einer komplett homogenen Integration der Website mit der Software, da sowohl ASP.NET, als auch C\# die Windows .NET-Library benutzt. Zur Kommunikation kann \gls{wcf} WCF (Windows Communication Foundation), eine Sammlung von Kommunikationswerzeugen zwischen verteilten Systemen innerhalb von .NET. Au�erdem ist, angenommen man bleibt in der Windows-Welt, sowohl die Implementierung, als auch die Separation of Concerns (SoC) einfacher umzusetzen. Mit Hilfe von \gls{wcf} kann der Aufruf von entfernten Methoden sogar �ber JavaScript geregelt werden, um eine noch st�rkere logische Trennung zwischen Software- und Websitefunktionalit�t zu schaffen.\\
	Die Chartingfunktionen k�nnen seit .NET-Version 3.5 direkt mit dem inkludierten Framework "Chart Controls for .NET" und dem ASP-Chart-Tag umgesetzt werden. Das Framework unterst�tzt unter anderem auch "Advanced Financial Charts". \footnote{url{http://www.microsoft.com/en-us/download/details.aspx?id=11001\#overview}}\\
\\
Die Variante mit Java Servlets und JSP ergab sich in der Nachforschung als ung�nstig, da zur homogenen Umsetzung die Software mit Java implementiert werden m�sste, was u.a. aus Performancegr�nden ausgeschlossen wurde. Ohne die Logikklassen in Java zu programmieren ergibt sich aus diesem Ansatz kein Vorteil.\\
\\
Das wahrscheinlich wichtigste Kriterium zur Wahl der Sprache ist die Kommunikation zwischen der C\# Software und der Website. F�r eine genauere Beschreibung der Datenschnittstellen siehe \ref{section:datenschnittstellen}.

\section{Datenschnittstellen} \label{section:datenschnittstellen}

Software <-> Website (Controlling)
Software -> Datenbank <- Website (Kursdaten)

* CGI (Ausf�hrung �ber CommandLine)
* Streams/Sockets
* Datenbank (Problem: Pull-Ansatz)
* WebServices
* WCF

* Entg�ltige Wahl der Websprache (ASP.NET)