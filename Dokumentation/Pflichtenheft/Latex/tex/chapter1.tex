% !TeX root = ../Aquila_Pflichtenheft.tex

% Chapter1
\chapter{Zielbestimmung} \label{chapter:zielbestimmung}
\section{Musskriterien}

Das Produkt erm�glicht das automatisierte Handeln von Aktien �ber einen bestehenden Online-Broker Account. Kursdaten, sowie eventuell andere relevante Daten, werden �ber einen Datenanbieter abgerufen und an das Algorithmus-Modul zur Verarbeitung weitergereicht.\\
	Die Steuerung und das Controlling der Software kann manuell �ber eine Webschnittstelle vorgenommen werden. �ber diese k�nnen die zu handelnden Aktien selektiert, sowie das einzusetzende Kapital festgelegt werden. Zus�tzlich soll dem Benutzer der Kursverlauf als Chart, sowie die aktuelle Entscheidung des Algorithmus und falls praktikabel die St�rke der Entscheidung angezeigt werden. Das Risiko ist je Aktie regelbar. Die Implementierung der Risikosteuerung kann durch eine der folgenden Methoden erfolgen:

\begin{itemize}
	\item{Regelung der H�he des Investitionskapitals}
	\item{Festsetzung der H�he der Cut-Losses-Schwelle}
	\item{Parametrisierung des Algorithmus betreffend Risiko \footnote{Sofern dies vom Algorithmus unterst�tzt ist}}
\end{itemize}

Der Zugang zur Website ist gesichert und durch eine Benutzerverwaltung, die gegebenenfalls mehrere Accounts zul�sst, geregelt.

\section{Wunschkriterien}

Die Webschnittstelle stellt neben den reinen Kursdaten zus�tzlich noch relevante \glspl{ma}, sowie eine Art der Support- und Resistance-Level, zur Bestimmung von �berkauften und �berverkauften Bereichen, dar, damit Entscheidungen des Algorithmus nachvollzogen werden k�nnen. Aus dem selben Grund wird sowohl die tats�chlich realisierte Performance �ber eine Zeitperiode, als auch der noch nicht realisierte Gewinn oder Verlust des aktuellen Trades (vorz�glich inkl. Transaktionsgeb�hren) angezeigt.\\
	Um die Entscheidungen besser nachvollziehen zu k�nnen oder auch um potentielle Fehlentscheidungen manuell korrigieren zu k�nnen, werden auf der Webseite aktuell relevante News-Headlines angezeigt.

% FUNKTIONEN:
%* Historische Entscheidungen des Algorithmus (Histogramm, Verlauf ODER Einstieg, Ausstieg)
%* Info �ber die Aktien
%* Automatik/Manuell
%* Liquidate (NOT-Aus)
%* Aktuelle Performance

\section{Abgrenzungskriterien}

Die bezogenen Kursdaten m�ssen, bedingt durch hohe Kosten, w�hrend der Entwicklungszeit nicht in Echtzeit bezogen werden. Allerdings muss darauf geachtet werden, dass die Umstellung f�r den tats�chlichen Einsatz problemlos vorgenommen werden kann.\\
	Der Algorithmus zum Entscheiden des richtigen Kauf- und Verkaufszeitpunkt wird in das fertige Produkt integriert. Dieser Algorithmus wird nicht in diesem Projekt entwickelt und ist nicht Vertragsgegenstand. Es wird hingegen eine Schnittstelle entwickelt und implementiert, um bestehende Algorithmen modular einzubinden.\\
	Die Zielgruppe des Produktes handelt mit relativ kleinen Positionen, weshalb Kaufentscheidungen weder in mehrere Orders geteilt (Oder-Splitting), noch auf Slippage (�nderungen im Aktienpreis durch eigene Order) R�cksicht genommen wird.\\
	Standardm��ig wird aufgrund der Datenkosten nur ein eingeschr�nkter Teil von Aktien unterst�tzt. Zus�tzliche Aktien k�nnen gegen Entgelt nachtr�glich hinzugef�gt werden.