\chapter{Technische Machbarkeit}\label{chapter:Technische Machbarkeit}
\section{Variantenbildung}
Die Tradingsoftware kann man in jeder erdenklichen Programmiersprache schreiben, allerdings ist es wichtig daran zu denken, dass das Programm einerseits effizient arbeiten soll und deswegen hardwarenahe rechnet, und andererseits hat das Projektteam mit manchen Programmiersprachen keinerlei Erfahrung.\\
Die allgemeine Funktionalit�t muss das Einlesen von Echtzeitdaten der B�rse und das Berechnen der Kaufentscheidung sein. F�r das Team kommen daher 3 M�glichkeiten in Frage: Eine L�sung in reinem C++, welches sehr hardwarenahe arbeitet, eine Mischung aus F\# und C\#, mit der eine parallelisierte Berechnung m�glich w�re, und eine reine F\#-L�sung. \\
Bei der Kombination agiert C\# als Handlungs- und Steuerkern, als auch zur Kommunikation mit der Website und dem News-Feed und F\# als funktionale Programmiersprache, als Rechenkern und \"Mastermind\" der Applikation, welche die Entscheidungen trifft. Hierbei wird einerseits eine enorm hohe Arbeitsgeschwindigkeit erm�glicht, da die beiden Sprachen relativ hardwarenah agieren und andererseits besteht der nicht zu untersch�tzende Vorteil bzw. die M�glichkeit, den Rechenkern auf ein externes System outzusourcen, welches zum Beispiel enorme Rechenkapazit�ten aufweisen k�nnte und somit viel komplexere und effizientere Algorithmen in annehmbarer Zeit durchrechnen und abh�ngig davon mehr gewinnbringende Entscheidungen treffen k�nnte. Dabei sollte es auch bei sp�teren Erweiterungen des Programms zu keinem signifikanten Geschwindigkeitsabfall kommen. Ein weiterer Vorteil einer F\# L�sung beziehungsweise einer Kombination w�re eine leicht auswechselbare Datei, die das Einf�gen der eigenen Algorithmen erleichtert und erm�glicht.


\begin{center}

\begin{tabular}{ | c | p{2.6cm} | p{1.7cm} | p{0.5cm} |p{0.5cm}|p{0.5cm}|p{0.5cm}|p{0.7cm}|p{0.7cm}|}
\hline 
\multicolumn{2}{|p{1.5cm}|}{ }  & Gewicht\-ung & \multicolumn{2}{p{1.5cm}|}{\textbf{C++ R*G}} & \multicolumn{2}{p{1.5cm}|}{\textbf{F\# R*G}} & \multicolumn{2}{|p{1.5cm}|}{\textbf{C\#F\# R*G}}\\ \hline
\multirow{6}{*}{Einfachheit} & Aufwand Coding & 10\% & 3 & 30 & 1 & 10 & 2 & 20 \\ \cline{2-9}
& Bedienung/ Wartung & 6\% &3&9&2&6&1&3\\ \cline{2-9}
& Update &3\%&3&9&2&6&1&3\\ \cline{2-9}
& Integration &5\%&3&15&2&10&1&5\\  \cline{2-9}
& Kenntnisse &6\%&3&18&1&6&2&12\\ \cline{2-9}
& \textbf{Gesamt}&30\%&3&90&2&44&1&46\\ \hline
\multirow{5}{*}{Leistung}& �bertragungs-zeit &6\%&1&6&3&18&2&12\\ \cline{2-9}
& Absturz\-sicherheit &5\%&1&5&2&10&3&15\\ \cline{2-9}
& Ressourcen-verbrauch &3\%&1&3&3&9&2&6\\ \cline{2-9}
& Datenumfang &1\%&1&1&3&3&2&2\\ \cline{2-9}
& \textbf{Gesamt} &15\%&1&15&3&50&2&35\\ \hline
\multirow{5}{*}{Kosten}& Lizenzen &10\%&1&10&1&10&1&10\\ \cline{2-9}
& Support &5\%&3&15&1&5&2&10\\ \cline{2-9}
& Betriebs-kosten &5\%&1&5&1&5&1&5\\ \cline{2-9}
& Dokumen\-tation &5\%&1&5&2&20&3&15\\ \cline{2-9}
& \textbf{Gesamt} &15\%&1&30&1&40&2&40\\ \hline
\multirow{4}{*}{Dokumentation}& Verf�gbarkeit &10\%&3&30&2&20&1&10\\ \cline{2-9}
& Voll\-st�ndigkeit &10\%&3&30&2&20&1&10\\ \cline{2-9}
& Qualit�t &10\%&2&20&1&10&1&10\\ \cline{2-9}
& \textbf{Gesamt} &30\%&3&80&2&50&1&30\\ \hline
\end{tabular}

\end{center} 

\begin{center}

\begin{tabular}{|l|r|p{0.8cm}|p{0.8cm}|p{0.8cm}|p{0.8cm}|p{0.8cm}|p{0.8cm}|} \hline
Kapitel&Gewichtung&\multicolumn{2}{p{1.8cm}|}{\textbf{C++}}&\multicolumn{2}{p{1.8cm}|}{\textbf{Java}}&\multicolumn{2}{p{1.8cm}|}{\textbf{C\#/F\#}}\\ \hline
Einfachheit&30\%&3&90&2&44&1&46 \\ \hline
Leistung&15\%&1&15&3&50&2&35 \\ \hline
Kosten&15\%&2&30&2&40&1&40 \\ \hline
Dokumentation&30\%&3&80&2&50&1&30 \\ \hline
\end{tabular}

\end{center}

\begin{center}
\begin{tabular}{|l|l|l|l|} \hline
Gesamtbewertung&&&\\ \hline
Endreihung &3&2&1\\ \hline
\end{tabular}
\end{center}

Aus der Nutzwertanalyse kann man entnehmen, dass die C\#/F\# Kombination als die favorisierte M�glichkeit ausgeht, weitere Vorteile die sich aus der Wahl dieser Mischung ergeben sind: gute Kenntnisse der Programmiersprachen, tolle Community und die Einfachheit, sowie die Erweiterbarkeit. Bei dieser L�sung wird die Steuereinheit vom C\# Teil des Programms �bernommen und die Rechenaufgaben werden von dem F\# Teil bearbeitet. Au�erdem ist das .net-Framework sehr beliebt, deswegen kann man damit rechnen das bei einem Problem gen�gend Helfer gefunden werden k�nnen.