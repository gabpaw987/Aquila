% !TeX root = ../Aquila_Pflichtenheft.tex

% Chapter7
\chapter{Benutzerschnittstelle} \label{chapter:benutzerschnittstelle}

Die Benutzeroberfl�che von Aquila besteht aus Website zur Steuerung und Information des Benutzers, die im Folgenden spezifiziert wird.\footnote{Die Bezeichnungen, sowie die genaue Gliederung der beschriebenen Bereiche ist vorl�ufiger Natur und k�nnte sich im Verlauf des Projektes g.g.f. noch �ndern.}\\
	Die Website hat ganz grob betrachtet 2 Bereiche. Einen f�r Einstellungen und Konfiguration, sprich s�mtliche Controlling-Aktionen, sowie einen f�r Darstellung und Information. Der Einstellungsbereich (\emph{Settings}) gliedert sich wiederum in allgemeine Einstellungen (\emph{General}), der z.B. Voreinstellungen verschiedener Tradingparameter erm�glicht, Auswahl und Konfiguration der zu Handelnden Aktien (\emph{Stocks}), sowie Account-Einstellungen zur Anmeldung bei der Website (\emph{Account}).

\begin{itemize}
	\item{Information}
	\item{Settings}
	\begin{itemize}
		\item{General}
		\item{Stocks}
		\item{Account}
	\end{itemize}
\end{itemize}

Besonders wichtig bei der Darstellung von Charts und Performance-Informationen ist ein �bersichtliches Layout. Au�erdem muss die gesamte Website eine �bersichtliche und rasch zu bedienende Navigation aufweisen, um m�glichst schnell auf Ereignisse reagieren zu k�nnen und m�glichst wenig Zeit mit der Interpretation der Daten zu verschwenden.\\
\\
/B10/\\
\emph{Aktie hinzuf�gen}\\
Unter dem Punkt \emph{Settings/Stocks} kann durch die Eingabe des Aktiensymbols eine Aktie vom Benutzer ausgew�hlt werden. Daraufhin muss der Benutzer die Gr��e der Postion einstellen und kann die Voreinstellungen zu der Cut-Loss-Schwelle, oder ob der Modus auf Automatik/Manuell gesetzt wird, ver�ndern. Bevor der Benutzer die Einstellungen best�tigt wird das Gesamtkapital angezeigt, dass f�r einen Trade aufgewandt wird.\\
\\
/B20/\\
\emph{Wechsel der Betrachteten Aktie}\\
Im Informationsbereich kann der Benutzer aus den Aktien, die er aktuell handelt, eine ausw�hlen, f�r die anschlie�end alle Informationen und Charts angezeigt werden.\\
\\
/B30/\\
\emph{Anzeige der Performance}\\
Die Performance des Algorithmus wird in Zahlen f�r die betrachtete Aktie f�r eine bestimmte Zeitperiode angezeigt.\\
\\
/B40/\\
\emph{Charts}\\
Zur Information des Benutzers soll als Chart die Preisentwicklung der betrachteten Aktie, sowie die \glspl{ma} und sowohl Einstiegs-, als auch Ausstiegspunkte des Algorithmus angezeigt werden. Dies soll sowohl kurz-, als auch langfristig m�glich sein.\\
Au�erdem soll das aktuelle Ergebnis des Algorithmus angezeigt werden.