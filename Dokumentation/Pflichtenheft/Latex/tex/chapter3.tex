% !TeX root = ../Aquila_Pflichtenheft.tex

% Chapter3
\chapter{Produktumgebung} \label{chapter:Produktumgebung}

\section{Software}
\label{software}

Um das Ausf�hren der Software zu gew�hrleisten, ist es n�tig einen Webservice anbieten zu k�nnen. Weil das Teilprodukt, die Website, haupts�chlich mit ASP.NET geschrieben ist, muss einer der folgenden Webserver lauff�hig installiert sein:
\begin{itemize}
	\item \textbf{IIS} (Internet Information Services)
	\item \textbf{Apache-Webserver} (mit "`mod\_aspdotnet"' und "`mod\_mono"') 
	\item \textbf{XSP-Webserver}
	\item \textbf{Cassini-Webserver}
\end{itemize}
Im Falle, dass ein Linuxserver f�r die Website benutzt werden soll, wird ein Apache oder der XSP-Webserver empfohlen, damit es zu keinen schwerwiegenden Problemen kommt.
\\
Es wird angeboten, Website und Software auf unterschiedlichen Computern zu installieren und dort zu benutzen.
F�r das reibungslose Integrieren der Software wird ben�tigt:
\begin{itemize}
	\item ein Datenbankserver zum Interagieren mit der Website (Postgresql)
	\item ein installierter und funktionierender \textbf{\gls{ib}-Client}
	\item ein Client von e-Signal
	\item die aktuellste Version des \textbf{.NET-Frameworks}
\end{itemize}

\section{Hardware}
\label{hardware}

Im Falle \textbf{eines} Servers ben�tigt der Rechner auf jeden Fall 4 GB Arbeitsspeicher, zirka 100GB freien Speicher und zumindest einen Dual-Core Prozessor.\\
Allerdings liegen diese Werte bereits unter dem heutigen Standard f�r normale Server. Wenn man die mindeste Konfiguration w�hlt, wird empfohlen, keine weiteren Servert�tigkeiten �ber dieses Ger�t zu vollziehen. Es ist au�erdem eine Internetanbindung erforderlich, sowie ausreichend viel Speicherplatz auf der/den Festplatte/n.

\section{Orgware}
\label{orgware}
Der Server, auf dem die Software und die Website verwendet werden, muss mit dem Internet verbunden sein, damit er sich mit \gls{ib} und e-Signal verbinden kann. Dazu wird zuz�glich ein Account dieser beiden Anbieter ben�tigt.