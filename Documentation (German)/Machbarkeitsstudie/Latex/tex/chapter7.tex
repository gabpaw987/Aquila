\chapter{Management Summary} \label{chapter:management_summary}

Nach ausf�hrlicher Besch�ftigung mit den kritischen Themen sind f�r das Projekt AQUILA geeignete L�sungswege entwickelt worden und es wurde aus mehreren potentiellen Varianten die g�nstigste ausgew�hlt. Folglich ist das Projekt durchf�hrbar.\\
\\
Aus der Variantenbildung ergab sich die Wahl der Programmiersprachenkombination C\#/F\# f�r die Software, da dadurch die Performance der Programmierung und der Ausf�hrung gleicherma�en gegeben ist. Die Webschnittstelle soll in ASP.NET implementiert werden, da dadurch die Funktionen der .NET-API weitergehend verwendet werden k�nnen. Weil beide Technologien aus einer Hand kommen, kann \gls{wcf} f�r die Schnittstelle zwischen beiden Komponenten verwendet werden.\\
\\
Die Gesamtkosten des Projektes wurden auf \EUR{44381} veranschlagt. Empfohlene Preise f�r die Software liegen bei \EUR{1999} f�r eine Einzelbenutzerlizenz, \EUR{4999} f�r eine 5-Benuter Lizenz und \EUR{9999} f�r eine unbeschr�nkte Lizenz.\\
\\
Der gesch�tzte Aufwand des Projektes betr�gt 213.3 Stunden, der auf 3 Projektteammitglieder aufgeteilt wird. Der geplante Projektzeitraum ist vom 14.11.2012 bis 10.04.2013.
