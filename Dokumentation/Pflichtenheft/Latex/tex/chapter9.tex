\chapter{Globale Testf�lle}
\label{chapter:Globale_Testf�lle}

/T010/ \textbf{Angabe eines falschen Accounts} \\
Wenn ein Benutzer auf der Website versucht sich mit einem nicht vorhandenen Benutzernamen oder einem falschen Passwort anzumelden soll es ihm nicht m�glich sein, weitere Einstellungen oder Ver�nderungen am System vorzunehmen.
\\ \\
/T020/ \textbf{B�rse hat geschlossen} \\
Wenn dieser Fall eintritt soll weiterhin die M�glichkeit bestehen, auf der Website Einstellungen und Ver�nderungen vorzunehmen, wenn diese die Software betreffen, wird diese, diese �nderungen �bernehmen und bei Ablaufstart anwenden.
\\ \\
/T030/ \textbf{Beenden des Internetbrowsers w�hrend einer Session}\\
Wenn jemand gerade Einstellungen trifft und sein Internet Browser wird geschlossen oder reagiert nicht mehr sollen keine der unfertigen oder unvollst�ndigen �nderungen in Kraft treten damit es auf keinen Fall zu inkonstistenen Systemzust�nden kommen kann. Dies soll dann nicht im Logfile ersichtlich sein.
\\ \\
/T040/ \textbf{Empfangen der Einstellungen von der Website} \\
Wenn die Website Befehle erh�lt, die Einstellungen der Software zu �ndern sollen diese �nderungen an die Software �bertragen  und dann �bernommen werden. Dies soll im Logfile ersichtlich sein.
\\ \\
/T050/ \textbf{Beenden der Software} \\
Wenn die Software beendet wird, soll darauf geachtet werden, dass zum Zeitpunkt des tats�chlichen Beendens keine Transaktion mehr abgewickelt wird und das die zuletzt durchgef�hrte Transaktion bereits in den Log �bernommen wurde und an die Datenbank weitergereicht wurde. Dies wird ben�tigt damit man sich beim beenden der Applikation sicher sein kann, dass auf jeden Fall keinerlei Daten verloren gehen.
\\ \\
/T060/ \textbf{Daten empfangen, Log bearbeiten und Entscheidung speichern} \\
Nach dem Start der Software sollen Daten empfangen werden, diese Daten werden dann in den Algorithmus eingespeist um mit ihnen zu rechnen. Das Ergebnis dieses Prozesses ist eine Entscheidung. Diese kann man im Log, in der Datenbank oder auch auf der Website nachvollziehen.
\\ \\
/T070/ \textbf{Charts anzeigen} \\
Auf der Website soll es m�glich sein Charts zu den aktuellen Berechnungen zu sehen, zus�tzlich kann man auf der gleichen Seite auf der sich die Charts befinden, auch die letzte Entscheidung des Algorithmus betrachten. Dadurch erkennt man die Konnektivit�t zwischen Website und Software.