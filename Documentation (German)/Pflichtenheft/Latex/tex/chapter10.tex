\chapter{Entwicklungsumgebung}\label{chapter:Entwicklungsumgebung}

\section{Software}

\begin{itemize}
	\item \textbf{Microsoft Office Suite}\\
Die Office-Produkte von Microsoft haben in diesem Projekt eine hohe Bedeutung. Bei nahezu jeder Art von Dokumentation werden die Programme Word bzw. Excel dieser Produktreihe verwendet. Sie bieten einen hohen Komfort im Umgang mit Dokumenten und sind bei den Projektmitgliedern bereits �ber Jahre hinweg zur Gewohnheit geworden.
	\item \textbf{Astah}\\
Astah findet vor allem in der Planungsphase des Projekts seinen Nutzen. Denn die UML-Diagramme, die in der Planung zur Programmierung erstellt werden m�ssen, k�nnen damit sehr einfach und sch�n dargestellt werden.
  \item \textbf{Microsoft Visio}\\
Microsoft Visio, das zus�tzlich zur standardm��igen Ausstattung der Office-Suite installiert werden kann, findet bei diesem Projekt auch seine Verwendung. Es wird haupts�chlich verwendet, um die ER-Diagramme f�r die Konstruktion der Datenbank zu zeichnen, die die Erstellung der Datenbank deutlich vereinfachen.
	\item \textbf{Microsoft Visual Studio}\\
Microsoft Visual Studio ist eigentlich die meistben�tzte Software bei diesem Projekt. Es dient als Entwicklungsumgebung, die alles verwaltet, was im Laufe des Projekts an C\#- und auch F\#-Code anf�llt. Au�erdem wird die Dokumentation des Codes in Microsoft Visual Studio vorgenommen.
	\item \textbf{GIT-Tool}\\
Um auf den verwendeten GIT-Server zugreifen zu k�nnen, ist ein GIT-Tool n�tig. Da sich Aquila ausschlie�lich auf der Windows-Ebene bewegt und ein starkes aber dennoch einfaches GIT-Tool ben�tigt, fiel die Entscheidung auf SmartGIT. Dieses Tool ist f�r die kommerzielle Nutzung leider kostenpflichtig. Da dieses Projekt allerdings keinen direkt kommerziellen Nutzen haben soll, kann es ganz einfach kostenlos benutzt werden. Falls es allerdings dennoch notwendig sein sollte, unter einem anderen Betriebssystem zu arbeiten, so gibt es auf jeder anderen Plattform auch reichlich andere Alternativen.
\end{itemize}

\section{Hardware}

\begin{itemize}
	\item \textbf{Global verf�gbarer Server (GIT)}\\
Ein Server wird ben�tigt, um darauf GIT laufen zu lassen. Dieser sollte nat�rlich global verf�gbar sein. �ber Passwort und Usernamen k�nnen sich dann die Projektmitglieder mit diesem verbinden und die neuste Version des Quellcodes beziehen bzw. das Repository am GIT-Server aktualisieren. 
Dadurch wird ein Arbeiten im Team vereinfacht und optimiert, indem die Mitglieder immer den aktuellen Quellcode zur Verf�gung haben. Die Authentifizierung am Server gew�hrleistet nat�rlich den Schutz vor unbefugtem Zugriff.
  \item \textbf{Computer f�r die Mitglieder}\\
F�r den Fortschritt des Projekts und die Entwicklung des daraus entstehenden Systems ist f�r jedes Mitglied ein Computer bzw. Laptop notwendig. Diese Ger�te m�ssen nat�rlich einen fl�ssigen Umgang mit der im Projekt verwendeten Software gew�hrleisten.
  \item \textbf{Weiterer Server}\\
Da sich manche User aus Sicherheitsgr�nden w�nschen, den Application-Server und den Webserver, die bei der Verwendung des Produkts Aquila ben�tigt werden, auf unterschiedliche Rechner aufzuteilen, ben�tigen wir zu Testzwecken einen zus�tzlichen Server. Dieser kann allerdings auch auf einem anderen PC aufgesetzt werden, also werden hierf�r zum Testen keine Kosten anfallen.
\end{itemize}

\section{Orgware}

\begin{itemize}
	\item \textbf{UML (Unified Modeling Language) }\\
Durch die Verwendung der standardisierten UML f�r Diagramme (und �hnliches) wird die Kommunikation im Team optimiert und es werden Unklarheiten beseitigt.
  \item \textbf{.Net Framework  Library}\\
Bei etwaigen Problemen mit bestimmten Strukturen von C\#/F\# wird die offizielle Library verwendet.\\ Diese findet man unter http://msdn.microsoft.com/en-us/library/.
  \item \textbf{Richtlinien zur Codierung}\\
Um den Code zu vereinheitlichen und die Einlesezeit zu minimieren, werden gemeinsame Richtlinien f�r die Codierung geschaffen. Diese bestimmen, wie Variablen oder Methoden zu benennen sind und definieren die Struktur des Codes.
\end{itemize}
