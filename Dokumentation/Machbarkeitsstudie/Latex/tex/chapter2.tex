% Chapter2

\chapter{Voruntersuchung} \label{chapter:voruntersuchung}

\begin{quotation}
``The improvement of \textbf{understanding} is for two ends: first, our own increase of knowledge; secondly, \textit{to enable us to deliver} that knowledge to others.``
\begin{flushright}
(John Locke)
\end{flushright}
\end{quotation}

\section{IST-Erhebung}

Der Wertpapierhandel erf�hrt eine zunehmende Professionalisierung, was dazu f�hrt dass sowohl professionelle Trader in gro�en Unternehmen, als auch semiprofessionelle mit immer gr��eren Datenmengen, mehr Informationen und durch schnellere M�rkte gleichzeitig k�rzeren Entscheidungdauern konfrontiert werden.\\
Um dem Trend zu programmatischen Trading-L�sungen entgegenzukommen gibt es eine Reihe von Datenanbietern, die Kursdaten in unterschiedlichen Zeitabst�nden und mit unterschiedlicher Aktualit�t anbieten. Je aktueller und �fter ein Tick, also ein Kursstand zu einem Zeitpunkt, desto teurer ist die Datenanbindung im Durchschnitt. Einer der f�hrenden Echtzeit-Marktdaten Anbieter ist eSignal \footnote{http://www.esignal.com/default.aspx?tc=},
die eine Selektion an Produkten zu verschiedenen Konditionen anbieten.
\footnote{Preise s. http://www.esignal.com/esignal/pricing.aspx?tc=}
Die Version der Software "eSignal OnDemand" bietet die M�glichkeit Intraday-Kursdaten zu g�nstigeren Konditionen 15 Minuten verz�gert abzurufen. F�r einen Produktivbetrieb ist diese Verz�gerung vermutlich zu lange, aber sicherlich hinreichend f�r den Entwicklungs- und Testbetrieb.\\
Die Anbindung des Datenproviders kann �ber \acronym{dde} erfolgen.\\
\\
Um eine individuelle Trading-Strategie zu verwirklichen werden oft Online-Broker verwendet,
�ber die resultierende Orders direkt mit vergleichsweise niedrigen Spesen abgewickelt werden k�nnen
und algorithmisches Trading �berhaupt erst praktisch m�glich wird. \gls{ib}
\footnote{http://www.interactivebrokers.com/ibg/main.php} ist ein erw�hnenswertes Beispiel eines solchen.
F�r \gls{ib} ist bereits ein Test-Account vorhanden, �ber den Transaktionen virtuell durchgef�hrt werden k�nnen.
\\
Im Bereich des algorithmischen Tradings sind bei high-performance Anwendungen C, C++ und deren Derivate sehr beliebt, bei Anwendungen, die nicht innerhalb von Sekundenbruchteilen operieren wird die .NET Umgebung unter Windows h�ufig verwendet.\\
Hinsichtlich stehen innerhalb des Projektteams mehrere Arbeitspl�tze mit .NET-Umgebung zur Verf�gung.

\section{IST-Zustand}

Wie bereits erw�hnt wurde, ist das algorithmische Handeln eine riesige internationale Industrie. Die meisten Banken haben in der Investment-Abteilung Projekte im Bereich der Entwicklung von Algorithmen zum Wertpapierhandel. Gut funktionierende Algorithmen werden in der Regel aber geheim gehalten, um sich einerseits nicht in die Karten blicken zu lassen und den institutionellen Gewinn zu optimieren. Sollte die Handelsstrategie publik werden, w�re diese leicht auszunutzen und Schwachstellen zu finden.\\
Gro�e Institutionen, wie Banken sind aber l�ngst nicht die einzigen, die solche Programme nutzen und entwickeln. Mit der zur Verf�gung stehenden Technologie kann heutzutage jeder mit Programmierkenntnissen und dem n�tigen finanzwirtschaftlichen Wissen Trading-Software entwickeln.
Das f�hrt dazu, dass Fonds und selbst private Investoren diese Schiene des Investment nutzen.\\
\\
Neben der propriet�ren Software gibt es auch einige wenige Anbiter, die teilweise Handelsentscheidungen im Abo verkaufen, Managed-Account L�sungen, denen der Zugriff auf das Investment-Portfolio gestattet wird und sogar \glsp{ide}, die eine Broker-Schnittstelle zur Verf�gung stellen und wo der verwendete Algorithmus selbst programmiert werden kann.

\section{SOLL-Zustand}